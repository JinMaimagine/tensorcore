\documentclass[zihao=-4, UTF8]{ctexart}
\usepackage[a4paper,margin=2.5cm]{geometry}
\usepackage{graphicx,subcaption}
\usepackage{float}
\usepackage{amsmath,amssymb}
\usepackage[colorlinks,linkcolor=blue,citecolor=blue]{hyperref}
\usepackage{enumitem}

% 自定义插图命令
\newcommand{\myfig}[4][]{
  \begin{figure}[H]
    \centering
    \includegraphics[width=#2\textwidth]{figures/#3}
    \caption{#4}
    \label{fig:#3}
  \end{figure}
}

\title{架构说明与分工明细}
\author{马晋}
\date{\today}

\begin{document}
\maketitle
\tableofcontents
\clearpage

% 各模块
\section{分工}
\subsection{马晋}
\begin{itemize}
  \item 完成了基本的MAC模块(参考了多种版本FP的比较:排除了FPnew,因为FPnew冗余过多及实现简单,pipeline方式低效)
  \item MAC模块FP32的pipeline+MAC模块中的加法部分(同时支持4bit和8bit)+FP16toFP32.sv
  \item transformtoAXI.sv+addrgen.sv+systolic.sv+control.sv+ENABLE.sv+tensorcore.sv
  \item 对整体的tensorcore进行了测试以及debug,包括找到部分MAC的错误
  \item 全局统筹项目
\end{itemize}
\subsection{李亚飞}
\begin{itemize}
    \item 完成了tensorcore中burst\_num和burst\_size以及systolic\_time和waitwrite\_time的计算并加入
    \item 完成了各种axi的逻辑,以及最后在PE中加入了写回到AXI的逻辑,并组装成整体的模块
    \item 完成了FP16toFP32模块+FP32toFP16.sv从而组合出MAC\_FP
    \item 组装了MAC\_FP和MAC\_adder从而完成顶层MAC,同时支持多精度FP,INT
    \item 对整个MAC模块进行了测试以及debug
    \item 协调统筹项目
\end{itemize}
\subsection{王俊涵}
\begin{itemize}
    \item 完成了mult\_INT4(自写testbench),mult\_INT8(自写testbench),CLA(自写testbench),MAC(testbench由马晋提供)的验证
    \item 完成了MAC的综合+最终top的综合
\end{itemize}
\subsection{补充}
\begin{itemize}
    \item 马晋和李亚飞参与了初步的架构构建,在这个过程中马晋提出了采用分块乘法并在最后阶段做加法的想法,两人确定了其中部分参数
    \item 马晋自己完成了最终的架构构建:考虑了数据排布的影响,否决了前面提出的矩阵乘法的顺序
    \item 在最终的架构构建阶段,考虑将systolic与PE的复杂性解耦合,用addrgen专门生成数据取地址,大大提高了效率
    \item 马晋可能会考虑继续对于架构进行优化,目前在INT4与INT8的情况下,算力>>带宽,但是很容易将架构进行扩展,利用多个SRAM使得算力得到充分运用
\end{itemize}
\subsection{贡献度:仅供参考}
\begin{itemize}
    \item 马晋: 70\%
    \item 李亚飞: 25\%
    \item 王俊涵: 5\%
\end{itemize}
\end{document}